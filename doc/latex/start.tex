\hypertarget{start_faq}{}\section{Small F\-A\-Q}\label{start_faq}
\hypertarget{start_description}{}\subsection{What is this library?}\label{start_description}
This library allows development on a L\-P\-C17xx board from N\-X\-P. The release includes configuration for the M\-B\-E\-D platform which uses a L\-P\-C1768.\hypertarget{start_needs}{}\subsection{What is needed to use rf\-L\-P\-C?}\label{start_needs}
In order to compile, you will need a gcc version tuned to build bare metal binaries. I use the one which can be build using the shell script available here\-: \href{https://github.com/esden/summon-arm-toolchain}{\tt https\-://github.\-com/esden/summon-\/arm-\/toolchain}

This release also includes two header files from A\-R\-M and N\-X\-P that defines C structures to access Cortex M3 and L\-P\-C17xx registers.\hypertarget{start_inside}{}\subsection{What is provided?}\label{start_inside}
This library provides
\begin{DoxyItemize}
\item A linker script
\item Board initialization code (mainly, data and bss segment initialization, moving the interrupt vector in R\-A\-M and configuring the P\-L\-L0 to set the C\-P\-U clock to something usable (96\-Mhz for the M\-B\-E\-D)
\item Few libc like functions (memcpy, printf)
\item Drivers for several peripherals
\begin{DoxyItemize}
\item Onboard L\-E\-Ds
\item Ethernet 100\-Mbps
\item G\-P\-I\-O
\item Repetitive Interrupt Timer
\item System Tick Timer
\item General purpose Timers
\item U\-A\-R\-T 
\end{DoxyItemize}
\end{DoxyItemize}\hypertarget{start_missing}{}\subsection{What is missing?}\label{start_missing}
A lot of stuff ! Mainly the remainder of drivers for the other devices such as
\begin{DoxyItemize}
\item C\-A\-N
\item D\-A\-C/\-A\-D\-C
\item P\-W\-M
\item ... 
\end{DoxyItemize}\hypertarget{start_use}{}\subsection{What is using this library ?}\label{start_use}
This library has mainly been developped for two purposes
\begin{DoxyItemize}
\item Enjoying myself
\item Provide a nice prototyping platform for my research team (you can have a look at my research work here \href{http://www.lifl.fr/~hauspie}{\tt http\-://www.\-lifl.\-fr/$\sim$hauspie})
\end{DoxyItemize}

For this last purpose, the M\-B\-E\-D was a nice and affordable platform although the whole on the cloud compiler stuff was not much what pleases us as the libraries provided by mbed were H\-U\-G\-E and we wanted to release A\-L\-L our code open source, even the low level code. The library provided by N\-X\-P (C\-M\-S\-I\-S) was a bit too high level for what we wanted and thus, writing everything from scratch was our best option.

The first of our project which has been ported to the M\-B\-E\-D is Smews\-: Smart \& Mobile Embedded Web Server (\href{http://www.lifl.fr/2XS/smews}{\tt http\-://www.\-lifl.\-fr/2\-X\-S/smews})\hypertarget{start_how}{}\section{Environment configuration and building the library}\label{start_how}
To use the library, you just have to compile it by issuing a {\ttfamily 'make'} in the main folder. It will build the library as well as all the samples. But before that, you have to install an arm compiler and modify the Makefile.\-vars to set the path and executable names of your compiler. The library should be shipped with the configuration for arm-\/none-\/eabi-\/$\ast$ tools suite. Modifying {\ttfamily P\-R\-E\-F\-I\-X} and {\ttfamily G\-C\-C\-\_\-\-V\-E\-R\-S\-I\-O\-N} should be enough.


\begin{DoxyCode}
 ## Modify these settings
 PREFIX=arm-none-eabi
 GCC_VERSION=

 ## System commands definitions
 CC=$(PREFIX)-gcc$(GCC_VERSION)
 LD=$(CC)
 AR=$(PREFIX)-ar
 AS=$(PREFIX)-as
 OBJCOPY=$(PREFIX)-objcopy
 OBJDUMP=$(PREFIX)-objdump
 NM=$(PREFIX)-nm
 SIZE=$(PREFIX)-size
\end{DoxyCode}


If everything builds, then you are ready to use it.

Otherwise, there are few things that you have to check\-:
\begin{DoxyItemize}
\item Do you have an arm compiler in your path?
\begin{DoxyItemize}
\item is the {\ttfamily Makefile.\-vars} file modified according to this compiler?
\end{DoxyItemize}
\item Have you modified the config/config file which has been generated when compiling the library for the first time?
\begin{DoxyItemize}
\item if so, then the library may compile but some samples will not depending on what functionalities you have enabled
\end{DoxyItemize}
\end{DoxyItemize}\hypertarget{start_first-program}{}\section{Your first program}\label{start_first-program}
The easiest way to start your first program is to copy the samples/skel folder and start from here.

In this folder, you will find two files
\begin{DoxyItemize}
\item Makefile
\item main.\-c
\end{DoxyItemize}

The Makefile rules how your program is compiled. Here is how it looks 
\begin{DoxyCode}
 OUTPUT_NAME=modify_this
 SRC=$(wildcard *.c)
 OBJS=$(SRC:.c=.o)
 
 # Modify this variable at your own risk
 RFLPC_DIR=../..
 include $(RFLPC_DIR)/Makefile.in
\end{DoxyCode}


\begin{DoxyItemize}
\item the {\ttfamily O\-U\-T\-P\-U\-T\-\_\-\-N\-A\-M\-E} variable define the name of the final binary. Here it will generate {\ttfamily modify\-\_\-this.\-elf} and {\ttfamily modify\-\_\-this.\-bin} files \item the {\ttfamily S\-R\-C} variable should contain the name of all your {\ttfamily }.c files. These files will be compiled and linked to the final binary \item the {\ttfamily R\-F\-L\-P\-C\-\_\-\-D\-I\-R} is the relative path from your folder to the folder that contains the {\ttfamily rflpc-\/config} file \item the {\ttfamily include} line includes the makefile that does all the magic for you. It contains generic rules for compiling your source files as well as the link rules to generate the {\ttfamily }.elf and {\ttfamily }.bin files\end{DoxyItemize}
Thus, you should just have to modify {\ttfamily O\-U\-T\-P\-U\-T\-\_\-\-N\-A\-M\-E} and {\ttfamily S\-R\-C} variables to create the needed makefile to compile your project.

To compile, just type {\ttfamily make}. You should see something like that


\begin{DoxyCode}
 $ make
 arm-none-eabi-gcc -mthumb -mcpu=cortex-m3 -fno-builtin -ffreestanding -Wall -
      Winline -O1 -I/home/hauspie/work/git/rflpc -DRFLPC_CONFIG_ENABLE_ATOMIC_PRINTF -
      DRFLPC_CONFIG_ENABLE_DMA -DRFLPC_CONFIG_ENABLE_ETHERNET -
      DRFLPC_CONFIG_ENABLE_MEMCPY -DRFLPC_CONFIG_ENABLE_MEMSET -DRFLPC_CONFIG_ENABLE_PRINTF -
      DRFLPC_CONFIG_ENABLE_PROFILING -DRFLPC_CONFIG_ENABLE_RIT_TIMER -DRFLPC_CONFIG_ENABLE_SETJMP -
      DRFLPC_CONFIG_ENABLE_SPI -DRFLPC_CONFIG_ENABLE_SYS_TICK_TIMER -
      DRFLPC_CONFIG_ENABLE_TIMERS -DRFLPC_CONFIG_ENABLE_UART -DRFLPC_CONFIG_PLATFORM_MBED   -c -o main.o 
      main.c
 arm-none-eabi-gcc -o modify_this.elf main.o -nostdlib -L/home/hauspie/work/git
      /rflpc/rflpc17xx -Wl,-T,rflpc17xx.ld,-Map=rflpc.map -lrflpc17xx 
 arm-none-eabi-objcopy -O binary -j .text -j .data modify_this.elf modify_this.
      bin
\end{DoxyCode}


If so, then you will have two files, an elf file and a bin file. The elf file is your program in E\-L\-F format. You can inspect it, dissassemble it... with commands such as your arm objdump. The bin file is the raw code memory file which is an extract of the {\ttfamily }.text and {\ttfamily }.data section of your elf file. For the M\-B\-E\-D platform, it is this file that you have to copy on the U\-S\-B mass storage drive.

To program your code on the M\-B\-E\-D, you can issue a {\ttfamily make program}. This command will try to guess the mountpoint of your M\-B\-E\-D (using the output of the {\ttfamily mount} command) and copy the bin file to it. After that, you just have to reset the M\-B\-E\-D to actually flash the code.\hypertarget{start_config}{}\section{Fine tuning the library}\label{start_config}
The library can be configured so that some features are not included. This can save loads of code memory when you just need a few drivers. \hypertarget{start_config-file}{}\subsection{Automatic generation of the configuration file}\label{start_config-file}
The configuration file is located in the config/config folder. When you clone the git repository, this file is N\-O\-T included. However, if you just use {\ttfamily make} in the library folder, a default full configuration file is generated

This file is a list of defines that will be enabled at compile time. The file is read by the {\ttfamily rflpc-\/config} script when generating the compile flags. Each line represents a define that will be transformed to a {\ttfamily -\/\-Dxxxx} flag.

To generate the file, you can use the makefile in the config folder. There are two main rules for generating a config file
\begin{DoxyItemize}
\item {\ttfamily make empty\-\_\-config}
\item {\ttfamily make full\-\_\-config}
\end{DoxyItemize}

The first one generates an empty config file. Then, when the library is compiled it is compiled with the minimum features which are\-:
\begin{DoxyItemize}
\item board initialization code
\begin{DoxyItemize}
\item clock configuration
\item bss and data segment initialization
\item default interrupt setup
\end{DoxyItemize}
\item interrupt management
\item Pins configuration
\item G\-P\-I\-Os
\item L\-E\-Ds
\end{DoxyItemize}

Pins configuration, G\-P\-I\-Os and L\-E\-Ds are only macros or inlines in the library. Thus, the produced code will only include it if you use it. The library in the minimal configuration is about 800 bytes of code.

The second option (full\-\_\-config) automatically extracts all the {\ttfamily R\-F\-L\-P\-C\-\_\-\-C\-O\-N\-F\-I\-G\-\_\-\-E\-N\-A\-B\-L\-E\-\_\-xxx} macros from the library source code and add it to the config file. Thus, all the functionalities of the library are included. At the moment, this produce a library that is about 8k\-B of code.\hypertarget{start_fine-tune}{}\subsection{Fine tuning the configuration file}\label{start_fine-tune}
The simplest way to fine tune the library is to start by a make full\-\_\-config and then remove the line you do not want from the config/config file. You can either remove the lines completely or use the {\ttfamily \#} character to make a line comment \char`\"{}à la\char`\"{} sh.

For example, this config file builds a library that uses only U\-A\-R\-T


\begin{DoxyCode}
 # You can use comments in the config file to disable a line or simply comment
 #RFLPC_CONFIG_ENABLE_TIMERS
 RFLPC_CONFIG_ENABLE_UART
\end{DoxyCode}
\hypertarget{start_warnings}{}\subsection{Common configuration mistakes}\label{start_warnings}
When fine tuning the configuration be sure to\-:
\begin{DoxyItemize}
\item Recompile the library A\-N\-D your program after making a change to the config file. (use make mrproper to clean the library and rebuild it)
\item Pay attention to dependencies. For example, if you enable printf but not U\-A\-R\-Ts, the default function used by printf to output its characters will do nothing
\item Most of the samples will not work if you do not enable at least uarts and printf 
\end{DoxyItemize}